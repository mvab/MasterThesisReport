Breast cancer is the second leading cause of death among women in developed countries. It is an extremely heterogeneous disease with distinct subtypes and clinical implications. One of the mechanisms through which cancer development could be controlled is autophagy. Autophagy is a highly-conserved process of cellular self-degradation in response to stress or nutrient starvation. It can promote both cell-survival and cell-death, hence it has been linked to many human pathologies including cancer. In cancer, autophagy is believed to have a complex and context-dependent role. At initial stages of cancer, it helps to recycle damaged molecules thereby reducing cytotoxic damage and preventing tumorigenesis. Impaired autophagy contributes to cancer progression. At later stages of cancer, tumour cells learn to use autophagy for their own cytoprotection and therey evade cancer therapies. 

However, the current understanding of the exact link between autophagy regulation and breast cancer development is very superficial and there are many unanswered questions. Both breast cancer and autophagy have seen a number of research highlights and breakthroughs in the past decade, therefore now it is time to start bridging the knowledge gap between the two by identifying autophagy signatures in breast cancer. 

In this project, The Cancer Genome Atlas breast cancer dataset, which is one of the largest publicly available datasets, was used to analyse gene expression changes between various breast cancer classification subgroups, including tumour morphology, cancer stage, and PAM50 molecular profile. The sample classification subgroups were extensively investigated with different exploratory analysis methods, which confirmed the tremendous heterogeneity of breast cancer. The exploratory analysis results were used to guide the differential expression analysis setup. Additionally, soft-clustering was performed to detect clusters of genes that have similar expression behaviour with respect to their changes along cancer stages. Following that, differentially expressed genes and the genes assigned to individual clusters were tested for autophagy enrichment. 

The enrichment analysis has shown that autophagy genes are overrepresented among the genes that are downregulated in cancer versus normal, regardless of subtype or morphology. This signature was found both in differential expression analysis and soft-clustering results. A consensus set of genes between the results of two methods was identified, thereby providing a list of candidate autophagy genes that can be further explored for their role in breast cancer. 
