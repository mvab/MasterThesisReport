\section{Data}
    \subsection{Data sources}
    
    The full TCGA-BRCA RNA-sequencing dataset was download from the GDC data portal using R/Bioincuctor package TCGAbiolinks v2.8.13
\cite{Colaprico2016}. Table \ref{table:full} shows the overview of samples collected by the TCGA research network. \\
    The original dataset has been subset to a more manageable collection of samples based of three main factors, i.e. include samples that are\textbf{ i)} female only (to reduce biological variation coming from gender) \textbf{ii)} present in the list of manually curated samples and their classifications \textbf{iii)} provided with enough metadata to benefit exploratory analysis. Only primary tumour and normal samples were included in the analysis.\\
    
    The curated lists of samples specifying stage and morphological type of samples were provided by the collaborators from other groups at DCRC.
    Additionally, a curated collection of autophagy-related genes was provided by experts in the field at DCRC (in the Cell Death and Metabolism and Cell Stress and Survival Units).\\
    
    The metadata for each sample was collated from three sources. The original clinical sample annotation extracted with TCGAbiolinks was complimented with the additional subtype classifications for previuosly unclassified samples from a recent TCGA-BRCA study by Ciriello \textit{et al.} \cite{Ciriello2015ComprehensiveCancer}. Further metadata was added from the work by Rahman \textit{et al. }\cite{RahmanAlternativeResults} on reprocessing TCGA data. \\   
    
        
    %   TABLE
            \begin{table}[!htbp]
                \centering
                \caption{TCGA-BRCA full dataset sample number overview}
                \label{table:full}
                    \begin{tabular}{c|c|c|c}
                    \textbf{Samples} & \textbf{Tumour} & \textbf{Normal} & \textbf{Metastasis} \\ \hline
                    \multicolumn{1}{|c|}{\begin{tabular}[c]{@{}c@{}}1212 \\ (F:1199, M:13)\end{tabular}} & \begin{tabular}[c]{@{}c@{}}1093\\ (F:1082, M;12)\end{tabular} & \begin{tabular}[c]{@{}c@{}}112\\ (F: 111, M:1)\end{tabular} & \multicolumn{1}{c|}{\begin{tabular}[c]{@{}c@{}}7\\ (F:7, M:0)\end{tabular}} \\ \hline
                    \end{tabular}
            \end{table}
            
            
    \newpage
    \subsection{Data quantification, extraction, and preprocessing}
    The original RNA-seq data was quantified by University of North Carolina (UNC) Center for Bioinformatics for the TCGA project \cite{UniversityofNorthCarolinaUNCCenterforBioinfromatics2013TCGAData}. The quantification pipeline of included  using Mapsplice v12.07 \cite{wang2010mapsplice} for mapping the data to reference genome (GRCh37/hg19), RSEM v1.1.13 \cite{li2011rsem} for transcript quantification. \\

    The data stored in TCGA data portal is accessible at three levels: 1) raw and uncontrolled, 2) normalised and controlled, and 3) integrated and interpreted. In this project, level 2 Legacy Breast Cancer dataset was downloaded and prepared using the TCGAbiolinks preprocessing pipeline. The pipeline contains integrated functions from the EDASeq package \cite{} for within-lane normalisation procedures to adjust for GC-content and gene length effects on read counts, as well as between-lane normalisation method to adjust for distributional differences between lanes (e.g sequencing depth) \cite{Colaprico2016, PapaleoTCGAPackages}, such as quantile normalisation (cut-off 0.10).  The pipeline transfroms the data into a '\textit{SummarizedExperiment}' \cite{Huber2015OrchestratingBioconductor} object (counts table), with genes and samples as rows and columns, respectively. \\
    
   \subsection{Sample annotation and gene curation}
   
   % about samples
   
   %so each sample is annotated with stage
   
   
   % about genes
   

% Please add the following required packages to your document preamble:
% \usepackage{graphicx}
% \usepackage[table,xcdraw]{xcolor}
% If you use beamer only pass "xcolor=table" option, i.e. \documentclass[xcolor=table]{beamer}
\begin{table}[!htbp]
\centering
\caption{My caption}
\label{my-label}
\resizebox{\textwidth}{!}{%
\begin{tabular}{lccclcclc}
\multicolumn{1}{c}{\textbf{Morphology}} & \textit{\begin{tabular}[c]{@{}c@{}}ICD-O-3 \\ code\end{tabular}} & \textbf{} & \textbf{} & \multicolumn{1}{c}{\textbf{Stage}} & \textbf{} & \textbf{} & \multicolumn{1}{c}{\textbf{PAM50 subtype}} & \textbf{} \\ \cline{1-3} \cline{5-6} \cline{8-9} 
\multicolumn{1}{|l|}{Lobular Carcinoma} & \multicolumn{1}{c|}{8520/3} & \multicolumn{1}{c|}{\cellcolor[HTML]{BBDAFF}100} & \multicolumn{1}{c|}{} & \multicolumn{1}{l|}{stage 1} & \multicolumn{1}{c|}{\cellcolor[HTML]{FFCCC9}100} & \multicolumn{1}{c|}{} & \multicolumn{1}{l|}{Luminal A} & \multicolumn{1}{c|}{\cellcolor[HTML]{FFFFC7}100} \\ \cline{1-3} \cline{5-6} \cline{8-9} 
\multicolumn{1}{|l|}{Infiltrating Duct Carcinoma} & \multicolumn{1}{c|}{8500/3} & \multicolumn{1}{c|}{\cellcolor[HTML]{BBDAFF}100} & \multicolumn{1}{c|}{} & \multicolumn{1}{l|}{stage 2} & \multicolumn{1}{c|}{\cellcolor[HTML]{FFCCC9}100} & \multicolumn{1}{c|}{} & \multicolumn{1}{l|}{Luminal B} & \multicolumn{1}{c|}{\cellcolor[HTML]{FFFFC7}100} \\ \cline{1-3} \cline{5-6} \cline{8-9} 
\multicolumn{1}{|l|}{Infiltrating Duct and Lobular Carcinoma} & \multicolumn{1}{c|}{8522/3} & \multicolumn{1}{c|}{\cellcolor[HTML]{BBDAFF}100} & \multicolumn{1}{c|}{} & \multicolumn{1}{l|}{stage 3} & \multicolumn{1}{c|}{\cellcolor[HTML]{FFCCC9}100} & \multicolumn{1}{c|}{} & \multicolumn{1}{l|}{Basal-like} & \multicolumn{1}{c|}{\cellcolor[HTML]{FFFFC7}100} \\ \cline{1-3} \cline{5-6} \cline{8-9} 
\multicolumn{1}{|l|}{Metaplastic carcinoma} & \multicolumn{1}{c|}{8575/3} & \multicolumn{1}{c|}{\cellcolor[HTML]{BBDAFF}100} & \multicolumn{1}{c|}{} & \multicolumn{1}{l|}{stage 4} & \multicolumn{1}{c|}{\cellcolor[HTML]{FFCCC9}100} & \multicolumn{1}{c|}{} & \multicolumn{1}{l|}{HER2-enriched} & \multicolumn{1}{c|}{\cellcolor[HTML]{FFFFC7}100} \\ \cline{1-3} \cline{5-6} \cline{8-9} 
\multicolumn{1}{|l|}{Mucinous adenocarcinoma} & \multicolumn{1}{c|}{8480/3} & \multicolumn{1}{c|}{\cellcolor[HTML]{BBDAFF}100} & \multicolumn{1}{c|}{} & \multicolumn{1}{l|}{stage X} & \multicolumn{1}{c|}{\cellcolor[HTML]{FFCCC9}100} & \multicolumn{1}{c|}{} & \multicolumn{1}{l|}{Normal-like} & \multicolumn{1}{c|}{\cellcolor[HTML]{FFFFC7}100} \\ \cline{1-3} \cline{5-6} \cline{8-9} 
\multicolumn{1}{|l|}{Other morphologies} & \multicolumn{1}{l|}{} & \multicolumn{1}{c|}{\cellcolor[HTML]{BBDAFF}100} &  & \multicolumn{1}{c}{} &  &  & \multicolumn{1}{c}{} &  \\ \cline{1-3}
\end{tabular}%
}
\end{table}

    
    \subsection{Finalised dataset}
    %table with all classifications, descripron of SE and sample matrix; other metadata available for majority od samples 
    
    
\section{Analysis methods}
    % paragraph avout EDA and how it is need before hypothesis-driven analysis

    \subsection{Principal Component Analysis}
    %what,why
    

    \subsection{Clustering and Heatmap representation}

    \subsection{Differential Expression Testing}

    \subsection{Soft clustering}
    
    \subsection{Enrichment Analysis}
    
    \subsection{PPi, TF, miRNA networks}