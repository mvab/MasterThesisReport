


\section{Scientific background}

    \subsection{Breast Cancer Overview}
    % Cancer is among the leading causes of death worldwide, as it takes hundreds of different forms depending on the location, cell of origin, and variety of genomic changes that promote oncogenesis and affect therapeutic response \cite{Weinstein2013TheProject}.
    
    % Breast cancer is the most common and fatal disease  among women in developed countries, with Denmark having the second highest disease incidence rate per 100,000 individuals in the world \cite{2015BreastInternational}. Breast cancer is not a single disease, but is composed of multiple subtypes with distinct morphologies and clinical implications \cite{Dai2015}. 
    
    
        
        % Growing evidence has implied that carcinomas with different histopathological and biological features exhibit distinct behaviours that lead to different treatment responses and require tailored therapeutic approaches \cite{Blows2010}. Therefore, accurate grouping of breast cancers into clinically relevant subtypes is of high importance for prognosis prediction and treatment choice making.
        % Until recently, immunochemistry (IHC) markers such as ER, PR, and HER2, along with traditional clinicopathological variables such as tumour size, tumour stage, nodal involvement, morphologic type were conventionally used for patient prognosis and management \cite{Vallejos2010}. However, with the wider adaptation of high-throughput platforms for gene expression, it has been shown that tumour cell response to treatment is determined by intrinsic molecular subtypes rather than physiological tumour factors \cite{Weigelt2010}. 

        \subsection{Management and prognosis}
        \subsection{Classification conventions}
            \subsubsection{Stages}
            \subsubsection{Morphology}
            \subsubsection{PAM50 molecular profile}
            % The original studies by Sørlie et al. (Sørlie et al. 2001) classified breast cancer tumours into five intrinsic subtypes with distinct clinical outcomes based on their ‘molecular portrait’: Luminal A, Luminal B, Basal-like, HER2-overexpressed, and Normal-like (Figure 1). The classification was guided by the differences underlying the gene expression patterns that reflect the fundamental differences of the tumours at the molecular level (Sørlie et al. 2003). The observed five subtypes almost perfectly mapped to the previously defined IHC subtypes (Dai et al. 2015), and have been repeated by several other studies with varying number of genes included in the subtypes’ signature. In 2009, Parker et al. (Parker et al. 2009) reported a clinically applicable 50-gene classifier, PAM50, containing mostly hormone receptor and proliferation-related genes. By comparing global gene expression data from microarray and qRT-PCR, a minimized set of 50 genes was identified that could reliably classify each tumour into one of the intrinsic subtypes with 93 accuracy. Over the past 7 years, the PAM50 intrinsic subtypes have shown to provide significant prognostic and predictive information beyond standard clinicopathological parameters (Vidal et al. 2017) (Gnant et al. n.d.). The PAM50 assay is now clinically implemented worldwide using the nCounter platform (Vidal et al. 2017). 

            \subsubsection{Other classifications based on gene expression }

        
        
    \subsection{The Cancer Genome Atlas}
        \subsubsection{TCGA goal and setup}
        % The Cancer Genome Atlas (TCGA) network (Anon n.d.), a part of Genomic Data Commons (GDC) from 2016, maintains a public database of clinical and molecular data over 38 different tumour types with hundreds of cases per type, making it the most comprehensive repository of human cancer data (Anon n.d.). Many important marker discovery papers were published with using TCGA data (Colaprico et al. 2016), but opportunities still exist to implement novel methods and test new hypotheses to characterise new biological pathways and diagnostic markers. 
        \subsubsection{TCGA breast cancer dataset}
        % The breast cancer samples available through the TCGA database, among other clinic- and histopathological annotations, are provided with the PAM50 label, allowing for rigorous analysis of the data and room for testing new hypotheses. 

        \subsubsection{TCGA breast cancer dataset previous work}
        
        
        
        
    \subsection{Autophagy}
    % In 2016, a Nobel Prize in Physiology or Medicine was given to Prof Yoshinori Ohsumi (Anon n.d.), a renowned scientist in the autophagy research field, for his success in elucidating the sophisticated machinery of the autophagy pathway. Because of his pioneering work, autophagy is recognized as a fundamental process in cell physiology with major implications for human health and disease. 
    % Autophagy is an evolutionarily conserved lysosomal degradation process that is crucial for adaptation to stress and maintaining cellular homeostasis (Feng et al. 2015). Recent studies have demonstrated association between autophagy and cancer, which implies that autophagy plays an important role in the development, progression, and response of breast cancer cells to chemotherapy and other therapies (Debnath n.d.). 
    
    % Upon starvation or stress conditions, autophagy is induced. Damaged organelles and cytoplasm are sequestered by an expanding phagophore, leading to the formation of double-membrane autophagosome (Feng et al. 2015), as shown in Figure 2. The autophagosome subsequently fuses with the lysosome, followed by the degradation of the sequestered cargo (Yorimitsu and Klionsky 2005). Then, the breakdown products are released back in cytosol through permeases. This allows recycling of the macromolecular constituents as building blocks, to generate energy to maintain cell viability under unfavourable conditions (Feng et al. 2013). 
    % Autophagy is involved in normal aspects of cell development and physiology, and defects in this process are associated with a range of diseases, including cancer. The work of Ohsumi and the colleagues has led to identification several dozens of autophagy-related genes, allowing for targeted research aimed at  understanding of the autophagy mechanisms, with the ultimate goal to manipulate or target them for therapeutic purposes (Feng et al. 2015). 


        \subsubsection{a}
        \subsubsection{b}


        
        
        
        
        
    \subsection{Autophagy in Cancer}
    % The role of autophagy in tumorigenesis and treatment response is complicated and context-dependent, and presumed to differ between stages of cancer progression (Zarzynska and Magdalena 2014). Autophagy’s role in maintaining organelle and protein turnover allows cells to restrain damage, including genome instability and inflammation, thereby limiting initiation and progression of cancer at early stages. However, once tumour develops, the cancer cells are able to utilize autophagy for their own cytoprotection (Zarzynska and Magdalena 2014). Autophagy is believed to promote cancer by allowing cells to survive under conditions of metabolic and genotoxic stress (Mathew and White n.d.). Unfavourable conditions such as hypoxia and acidity in tumour environment as well as the effects of chemo- and radiotherapy cause cells to experience stress and nutrient deprivation (Bailey et al. n.d.). Induction of autophagy within these cells allows them to survive the drastic conditions. This kind of mechanism had been referred to as “what doesn’t kill you, makes you stronger” by researchers in the field (Mathew and White n.d.).
        \subsubsection{a}
        \subsubsection{b}
        
        
        


\section{Thesis objectives}
% The role of autophagy in tumorigenesis and treatment response is complicated and context-dependent, and presumed to differ between stages of cancer progression (Zarzynska and Magdalena 2014). Autophagy’s role in maintaining organelle and protein turnover allows cells to restrain damage, including genome instability and inflammation, thereby limiting initiation and progression of cancer at early stages. However, once tumour develops, the cancer cells are able to utilize autophagy for their own cytoprotection (Zarzynska and Magdalena 2014). Autophagy is believed to promote cancer by allowing cells to survive under conditions of metabolic and genotoxic stress (Mathew and White n.d.). Unfavourable conditions such as hypoxia and acidity in tumour environment as well as the effects of chemo- and radiotherapy cause cells to experience stress and nutrient deprivation (Bailey et al. n.d.). Induction of autophagy within these cells allows them to survive the drastic conditions. This kind of mechanism had been referred to as “what doesn’t kill you, makes you stronger” by researchers in the field (Mathew and White n.d.).

