Understanding of autophagy role in cancer is becoming an increasingly intriguing area of research as the two constituting fields -- breast cancer and autophagy -- have seen a lot of progress individually in the past decade. This project aimed to identify the expression signatures of autophagy-related genes in the breast cancer transcriptomics data available from TCGA. 

Rigorous exploratory analysis was applied to the dataset to get an understanding of the main patterns characterising the available samples. Three main breast cancer classification methods that were used to stratify the patients into groups are cancer stage, tumour morphology, and PAM50 intrinsic molecular subtype. One of the main objectives of gene expression analysis was to identify a subset of samples described by those classification methods with a distinct autophagy-related signature. To do this, two complementary analysis methods were used -- differential expression testing and soft-clustering. 
In differential expression analysis, a multitude of contrasting tests were performed across different combinations of subgroups defined by classification methods, and the identified sets of differentially expressed genes were tested for autophagy enrichment. Soft-clustering aimed to detect clusters of genes that have similar collective expression behaviour with respect to their changes along cancer stages. Then, enrichment analysis was used to identify expression patterns where autophagy genes are overrepresented. Finally, a comparative analysis of two methods' results produced a list of autophagy genes that can be used as a starting point for further exploration of autophagy role in cancer.

Exploratory analysis of TCGA-BRCA dataset was a crucial part of this project. Firstly, it has confirmed the view that is now widely accepted in the breast cancer research community on the extreme heterogeneity of this disease, and the fact that cancer patients can most accurately be classified into prognostic groups by their intrinsic molecular profile rather than by tumour morphology and stage. The major differences between PAM50 subtypes were highlighted by PCA, hierarchical clustering, as well as differential expression testing, where individual subtypes had considerably more DEGs between each other than tumour morphologies or stages. In fact, neither exploratory analysis nor differential expression testing have found any strong differences between morphological groups or stages. The PCA plots for both did not reveal any PCs that describe the proposed sample subgroups sufficiently well. DE analysis found exceptionally low number of DEGs between cancer stages (even stage 1 and stage 4), and also not a great amount of DEGs between two major established morphologies, Lobular and Ductal carcinomas. This result was surprising, but the proposed explanation was in the underlying imbalanced composition of the subgroups. 

The imbalanced composition and sample size of the subgroups defined by classification methods became apparent in the exploratory analysis. The composition of subgroups in terms of the constituting PAM50 subtypes is important, as this classification method explains the main driving force that defines variation between samples. Therefore, there are two implications from this. First, if the groups are too mixed/heterogeneous (i.e. made up of various PAM50 subtypes), then it is difficult to capture differences between them, as the variation signal is too mixed and thereby weak. Second, if two groups that are being compared are exclusively made up of two different PAM50 subtypes, then the difference between them is likely to be driven by PAM50. A clear example of this was found when comparing Mucinous and Metaplastic morphologies, which are exclusively made up of Luminal and Basal samples, respectively. These two morphologies had the largest number of DEGs detected between them in comparison to the rest, in addition to being very clearly separated by PCA. However,  the true source of their variation is difficult to define. 
To tackle the problem of heterogeneity created by the strong influence of PAM50 profiles on other classifications, DEA with combined models was performed. It allowed to characterise differential expression between individual stages or morphologies within each PAM50 subtype, thereby creating more opportunities to test for autophagy signatures.

However, the results of enrichment analysis on DEGs were ambiguous and largely unsuccessful in identifying autophagy signatures in any of the specific sample groups. The main observed reocurring trend was the enrichment of autophagy-related genes among the downregulated genes in various cancer subgroups (including individual morphologies and stages) versus normal. Interestingly, there were no autophagy enrichment signatures between different stages of cancer, which is something that is often brought up in the literature to describe the behaviour of autophagy processes in cancer. Similarly, no significant autophagy enrichment was identified in the genes differentially expressed between the PAM50 cancer subtypes. Therefore, it was concluded that the main observed and statistically quantified autophagy signature in this breast cancer dataset is the overall downregulation of autophagy-related genes in cancer versus normal, regardless of stage or subgroup.
In line with that, soft-clustering analysis results have shown that the majority of autophagy genes are assigned to the cluster with expression pattern representing downregulation at all stages. This cluster pattern was found to be significantly enriched for autophagy, both at all genes or transcription factors-only level, and in separate PAM50 subtypes. 

Overall, it is important to highlight the significance of finding the trend of downregulation in cancer to be the main autophagy signature in this dataset. In a broad sense, it makes sense to see a subset of  autophagy-related genes to be downregulated in cancer. Deregulated autophagy means that the basal turnover of damaged macromolecules is impaired, which possesses an oncogenic impact. Additionally, seeing a group of autophagy genes downregulated consistently at all stages compared to normal, gives an indication of there being a part of the autophagy pathway that is inactivated is cancer in general. %more?

However, the literature on autophagy role in cancer often brings up the ‘dual role’ behaviour, which involves upregulation of  autophagy at different stages of cancer progression with opposite consequences (cytotoxicity and cytoprotection of cancer cells). The results of this project did not show any statistically significant indication of autophagy genes being upregulated at any specific stage or cancer in general. Although among the soft-clustering results there were several clusters of genes with potentially interesting expression patterns in relation to stage-specific upregulation,  the enrichment analysis did not indicate that those clusters possess significant autophagy signatures.

The inability to identify any autophagy signatures that would be in agreement with the literature-proposed expression patterns or that are cancer subtype-specific (as opposed to just cancer vs normal) and are significant (or significant at non-adjusted p-value level, as p-value cut-offs are arbitrary), leads to the discussion of the potential limitations of this study.
Firstly, it has to be reiterated that subgroup imbalance is very likely to be contributing to results both of exploratory and gene expression analyses. In addition to PAM50 composition influence discussed above, the number of samples in each  group is also of major concern. For instance, as in the current dataset there are only 12 samples of stage 4 (total $n=857$), and all of them are classified as one morphology, the conclusions made from analyses involving those samples have to regarded with caution. This is especially the case for when those samples are stratified by PAM50 subtype, as it results in only two samples representing the gene expression of Basal-like and HER2-enriched samples in differentially expression testing and soft clustering (see Table \ref{table:counts}). 

Hence, these dataset constraints may be the reason why no subtype-specific autophagy signatures were detected. Moreover, a possible interpretation of not seeing the literature-ascertained autophagy upregulation at later cancer stages that is related to metastasis initiation, is due to the fact that the primary tumour samples constituting this dataset are not a fit representation of cancer behaviour at later stages (even stage 4). Therefore, to see the true upregulation of autophagy as a result of cancer cells using it for their own benefit, the actual metastasis samples need to be analysed. 

\newpage
Another important aspect that was not taken into account in this project is the information on treatments that each patient has received. This information was not readily available at the time when it could have been utilised to benefit the project. However now, it is clear that having this information would have perhaps shed a light on the observed autophagy downregulation trend, or, rather the opposite -- the lack of upregulation, as some treatments can have activating or inhibitory effect on autophagy. 

In spite of the lack of subtype-specific signatures and overall dataset limitations, the comparative analysis of downregulated genes identified by two gene expression analysis methods was carried out to investigate the biological relevance of the putative autophagy signature. For each PAM50 subtype, a list of candidate genes that exhibit complete downregulation in cancer was produced. Among those genes, some interesting hits were identified, some of which are involved both in the key steps of autophagosome formation  and the signalling pathways that conduct the best-characterised autophagy induction/inhibition. 

\section{Future directions}

The main direction from the current phase of the project is to seek collaboration with the other units at DCRC who have the expert knowledge of autophagy and the related processes.  The identified list of candidate genes will be made available to them, and with their help, a more informed biological interpretation of the potential impact of downregulation of the identified genes can be achieved. Following that, particularly interesting genes can be sent for experimental validation in MCF7 breast cancer cell lines available at DCRC.

This is, however, not the end of the investigation of autophagy signatures in the breast cancer data. 
TCGA is a rich source of data, but the aforementioned constraints make drawing statistically significant conclusions from the results quite difficult. Therefore, the analysis should be  replicated on another, more balanced cohort of breast cancer patients. 

With the current results at hand, there are several directions where future analyses could go. Firstly, it would be interesting to explore the results of the differential expression analysis and soft-clustering separately. The results could be used as a guide for network analysis aiming to explore the relationship between deregulated genes and how their protein products interact together or modify each other, which will be possible by using the available experimental information on protein-protein interaction (PPI) networks deposited in databases (e.g. STRING). Additionally, retrieving information on the transcription factors and miRNAs that are known to regulate the differentially expressed autophagy genes could be used to analyse correlations in their own and their targets’ expression levels changes. It would be interesting to specifically identify those transcription modifiers that are altered in a certain breast cancer subtypes/stages/morphologies and for which the alteration is also reflected in the expression of the target autophagy genes. 

\section{Conclusion}


The over-arching observation made in this project is the presence of autophagy-related signature among the genes that are downregulated in cancer versus normal. The widely accepted view on autophagy being upregulated at different stages of cancer progression was not confirmed by the results of data analysis in this study.
The comparative analysis has helped to identify a set of autophagy-related genes that are downregulated at all stages in specific breast cancer subtypes according to both analysis methods. Obtaining this list of genes has opened a new avenue for research into the part of the autophagy mechanism that appears to be inactivated in cancer cells at all stages. A brief look at functions of the genes has shown that many of them are involved in the core steps of autophagy regulation, which is quite promising in itself. 

To conclude, understanding of the convoluted relationship between autophagy and cancer is still at very preliminary phase. When greater understanding of the signalling that both regulates and executes autophagy in cancer is achieved, targeting specific pathway components as a part of cancer treatment would maximise the benefit to breast cancer patients.






%  However, we are still at the very preliminary phase of understanding the intertwining relationship of autophagy and cancer. As we dig further, it gets clear that 
 
 
 
%  Targeting autophagy, particularly when it is acting in a survival mode, has significant potential to lead to the development of novel agents and therapeutic regimens. Existing

% Longer term success in targeting autophagy may require
% the development of a 



% Thus, a careful analysis
% of the full spectrum of effects of autophagy on cancer will be needed in order to successfully modulate it for therapeutic purposes. This may be a particular challenge for breast cancer therapy due to the high heterogeneity of this type of cancer but holds the opportunity for personalizing treatment protocols to maximize the benefit to breast cancer patients.
