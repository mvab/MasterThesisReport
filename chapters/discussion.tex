

% you will discuss how the hypothesis has been demonstrated by the new research and then show how the field's knowledge has been changed by the addition of this new data

%discussion starts with the interpretation of the results, then moves outwards to contextualize these findings in the general field.


%1. Begin with a restatement of your research question, followed by a statement about whether or not, and how much, your findings "answer" the question.  These should be the first two pieces of information the reader encounters.

%2. Relate your findings to the issues you raised in the introduction. Note similarities, differences, common or different trends.  Show how your study either corraborates, extends, refines, or conflicts with previous findings.


%3. If you have unexpected findings, try to interpret them in terms of method, interpretation, even a restructured hypothesis; in extreme cases, you may have to rewrite your introduction. Be honest about the limitations of your study.

\section{Combining DEA and clustering}
Paragraph 1: This paragraph provides a “big picture” perspective for readers to remind them of the importance of your study. 

Paragraph 2: This paragraph provides a critical analysis of your major finding(s). 
What was your overall approach for studying the gap? In one or two sentences, state the main models or strategies that you used to study this specific research question.

What was the most important result of your study? The focus of this paragraph is to highlight the most important contribution that your study has made. Explicitly state this result. Additional findings (major and minor) can be described in subsequent paragraphs. Do not repeat detailed results that can be found in the Results section. In general, specific figure numbers do not need to be re-stated in the Discussion unless you feel that doing so would substantially enhance your argument or discussion point. A schematic of your proposed mechanism or model can often be helpful for clearly and concisely summarizing your major result(s).

How does your result(s) fit with existing literature? This is an important part of the paragraph and may require multiple paragraphs depending on the number of key studies that exist on your topic. This paragraph should be well-rounded, meaning that contrary reports must also be discussed. In the case of a contrary report, you should state your interpretation of how and why the results of the two studies differed. For example, did the approaches differ or were there major differences in sample sizes that may have affected results? (Depending on how much information is available in the literature, a critical analysis of your major finding may require multiple paragraphs.)

Paragraph 3: Discuss additional findings and how these fit with existing literature.
 	Most studies yield multiple results. After you discuss your main result in the paragraph above, discuss additional major or minor findings. Unexpected and intriguing findings may be especially important to convey to readers. In addition, if a finding is contrary to what has been suggested in the literature, acknowledge this, and offer explanations based on your study. Even if a result was not statistically significant, it can be helpful to discuss a potential trend that may be important to assess in a future study. If these additional findings relate to your main finding, discuss the associations.

Paragraph 4: Discuss the limitations of the study.
 	Discuss potential limitations in study design. For example, how representative was your model? Did sample size affect your conclusions? Consider how these limitations affect the interpretation or quality of data. Do they affect the ability to generalize your findings?

Paragraph 5: Discuss future directions. 
What major follow up studies are indicated based on your results? Most studies yield new discoveries that prompt additional studies. Consider what new directions are supported by your findings. For example, do your experiments suggest that a specific molecule should be tested as a new drug target or that tissue-based studies or clinical investigations should be performed to translate your animal studies to patients? Making recommendations for follow-up studies is an important part of a Discussion.
Paragraph 6: Discuss your overall conclusion and the major impact of your study.
What is the main take-home message of your study?
What is the main contribution that your study makes to your field?
Relate this section to the first paragraph of the Discussion. In other words, how does your study fill “the gap” or address the problem that you presented in the Introduction and re-stated earlier in paragraph 1 of the Discussion?

\section{Conclusion}
%4. State the major conclusions from your study and present the theoretical and practical implications of your study.


\section{Future directions}

%5. Discuss the implications of your study for future research and be specific about the next logical steps for future researchers.