    
\section{Exploratory Analysis}

    %data was preprocessed as described in methods
    
    
    % The outcome of the EDA will have a major impact on how you analyze the data:
    %  – Sample outliers and noise: Should some samples be discarded?
    % Can I perform a meaningful test for different expression between different groups?
    % What factors (wanted/unwanted, observed/unobserved) should be included in the model?
    
    % • EDA is the first part of the analysis where you start drawing biological conclusions specific to a
    % given experiment.
    % – Does the data look like I expected/designed?
    % – Do I have batch effects?
    % – Subgroups?
    % • In DE you formalize your observations in the EDA to extract statistically valid results.
    % • The resulting DE analysis will serve as the foundation of biological interpretation of the data.

    
    

    \subsection{Main effect subgroups}

    There are three main effects by which breast cancer samples can be stratified into groups: morphology, stage and PAM50 molecular profile. PCA is able separate samples into tumour/normal sample clusters fairly well (shown as an example in Methods, Figure \ref{fig:pcamethod}). However, it is more interesting to apply PCA to explore variation by grouping samples in different cancer classification groups.  
    
    PCA was applied to 969 samples (857T, 112N), which were divided into subgroups by three systematic effect groups (i.e. PAM50, tumour morphology, cancer stage), which subgroup sizes shown in Tables \ref{table:morphstage} and \ref{table:pam50counts}. \\
    The PCA plot of data coloured by PAM50 molecular subtype groups is shown in Figure \ref{fig:pcapam50}. PC1 accounts for 11\% of the total variation in the data, and is driven by the differences between normal and cancer samples. PC2 accounts for 8.6\% of the variation, and characterises the variation among breast cancer subtypes. Firstly, Basal-like samples form a separate cluster (orange), which emphasises its complete dissimilarity at molecular level. Luminal A and Luminal B form a partially overlapping cluster (yellow and red), which is expected from the known similarities of these subtypes. HER2-enriched cluster (blue) is understandably located between Luminal B and Basal-like as it is known to share expression profiles with these two. And lastly, seeing Normal-like subtype (pink) overlapping with normal samples (green) as well as with Luminal A cancer subtype fits nicely with the fact that Normal-like subtype shares morphology with the former and IHC and molecular profile with the latter.    
    
    % PAM50 PCA plot 
            \begin{figure}[!h]
            \centering
            \includegraphics[scale=0.37]{pcapam502.png} 
            \caption{PCA plot showing the variance along PC1 and PC2 for PAM50 molecular subtypes of breast cancer and normal samples}
            \label{fig:pcapam50}
            \end{figure}
            
    \newpage
    One dimensional PCA plot based on PAM50 subtypes is shown in Figure \ref{fig:1dpcapam50}, where the variation along the first nine PCs is displayed as explained in the Methods section 2.2.1. Looking at the results in this way highlights again that PC1 is driven by tumour/normal variation. Together with PC2 they also capture the differences among PAM50 subtypes. As the PCs numbers increase, variation captured by them decreases. 
    
    % PAM50 1d PCA plot 
            \begin{figure}[!h]
            \centering
            \includegraphics[scale=0.5]{1dpcapam50.png}
            \caption{One-dimensional PCA plots showing variance along PC1-PC9 for PAM50 molecular subtypes of breast cancer and normal samples. }
            \label{fig:1dpcapam50}
            \end{figure}
    
    
    The PCA results for morphology and stages classifications are shown in Figure \ref{fig:1dpcamorphstage} as one-dimensional variation plots. It can be observed that PCA was not able to capture as much variation between morphology groups and stages as among PAM50 subtypes. From one-dimensional plots it is evident that PC1 captures cancer/normal differences in both classification types. However, the rest of PCs do not capture enough variation to be evident on 2D PCA plots, therefore, here only 1D plots are presented. Appendix \ref{} contains 2D PCA plots based on stages and morphology for results completion.    
    
    % 1D PCA for stages and morphology
            
            \begin{figure}[!h]
            \includegraphics[width=0.35\linewidth]{1dpcastages2.png}\hfill
            \includegraphics[width=0.35\linewidth]{1dpcamorph2.png}
            \caption{One-dimensional PCA plots showing variance along PC1-PC9 for morphology groups (right) and stages (left), with normal samples.}
            \label{fig:1dpcamorphstage}
            \end{figure}
            
    
    \newpage
    Interestingly, PCA was not able to capture any significant differences between stages (e.g. stage 1 and stage 4), where the differences in expression should be evident and therefore reflected in PCA results. The same is the case for two main distinct morphologies - Lobular and Ductal carcinomas, which overlap greatly. A possible explanation for these results is the unbalanced sizes of subgroups in these two classification methods, as well as their composition in terms of PAM50 subtypes, which is evidently the main driving force in sample classifications. 
    
    To explore this idea, the sample count data was visualised as stacked barplots presented in Figure \ref{fig:barms}. Each bar shows the total count of samples of a chosen morphology (left plot) or stage (right plot). The differing group sizes are evident.         
    
    % barsplots for stages and morphology 
       
        \begin{figure}[!h]
        \includegraphics[width=0.53\linewidth]{bar_morph.png}\hfill
        \includegraphics[width=0.4\linewidth]{bar_stages2.png}
        \caption{Stacked barplots of sample number counts per subgroups in morphology groups and stages classification groups. The coloured proportions represent PAM50 subtypes samples within each subgroup.}
        \label{fig:barms}
        \end{figure}
        
    Ductal carcinoma morphology samples make up over 70\% of all samples (?/857), but are well-proportionate in terms of PAM50 composition (i.e. similar proportions as the full dataset, with Luminal A dominating). Other subgroups, however, in addition to being smaller, are restricted to only a few PAM50 subtypes. For instance, Metaplastic carcinoma is made up completely of Basal-like samples, while Mucinous morphology is represented by only Luminal subtypes (the counts are shown in Table \ref{table:counts}). This observation explains why these two mophologies are separated in 1D PCA plot in PC2 (Figure \ref{fig:1dpcamorphstage}) - the difference between Luminal and Basal tissues is driving their variation. However, it is important to note, that there is a chance that these two morphologies, Metaplastic and Mucinous, actually only exists as Basal and Luminal subtype, respectively, but it is not possible to claim for or against this based on the current dataset.    
    
    The sample count difference between stages are also noticeable, which stage 2 accounting for roughly 50\% of the samples. The proportions of PAM50 subtypes within stages 1-3 appear to be well-balanced. Stage 4 is, however, of major concern. It has an alarmingly low sample count and is not represented in Normal-like subtype. The counts are shown in Table \ref{table:counts}.
        
        
    %TABLE x3 counts   
                \begin{table}[!h]
                \centering
                \caption{Samples number counts in pairwise comparisons of three main breast cancer samples classifications. Top table: PAM50 and stages, Middle table: PAM50 and morphology, Bottom table: stages and morphology. Each table shows row and column sums for each subgroup. The total count of samples in the project is in the lower right corner: 857.}
                \label{table:counts}      
                \resizebox{\textwidth}{!}{%
                \begin{tabular}{ccccccc}
                \multicolumn{1}{l|}{} & \multicolumn{1}{c|}{\textbf{Luminal A}} & \multicolumn{1}{c|}{\textbf{Luminal B}} & \multicolumn{1}{c|}{\textbf{Basal-like}} & \multicolumn{1}{c|}{\textbf{HER2-enriched}} & \multicolumn{1}{c|}{\textbf{Normal-like}} & {\color[HTML]{9B9B9B} \textit{rowsums}} \\ \hline
                \multicolumn{1}{c|}{\textbf{stage 1}} & \multicolumn{1}{c|}{94} & \multicolumn{1}{c|}{22} & \multicolumn{1}{c|}{21} & \multicolumn{1}{c|}{6} & \multicolumn{1}{c|}{5} & \multicolumn{1}{c|}{{\color[HTML]{656565} 148}} \\ \hline
                \multicolumn{1}{c|}{\textbf{stage 2}} & \multicolumn{1}{c|}{217} & \multicolumn{1}{c|}{99} & \multicolumn{1}{c|}{105} & \multicolumn{1}{c|}{47} & \multicolumn{1}{c|}{13} & \multicolumn{1}{c|}{{\color[HTML]{656565} 481}} \\ \hline
                \multicolumn{1}{c|}{\textbf{stage 3}} & \multicolumn{1}{c|}{93} & \multicolumn{1}{c|}{55} & \multicolumn{1}{c|}{21} & \multicolumn{1}{c|}{18} & \multicolumn{1}{c|}{8} & \multicolumn{1}{c|}{{\color[HTML]{656565} 195}} \\ \hline
                \multicolumn{1}{c|}{\textbf{stage 4}} & \multicolumn{1}{c|}{4} & \multicolumn{1}{c|}{4} & \multicolumn{1}{c|}{2} & \multicolumn{1}{c|}{2} & \multicolumn{1}{c|}{{\color[HTML]{C0C0C0} x}} & \multicolumn{1}{c|}{{\color[HTML]{656565} 12}} \\ \hline
                \multicolumn{1}{c|}{\textbf{unknown}} & \multicolumn{1}{c|}{12} & \multicolumn{1}{c|}{3} & \multicolumn{1}{c|}{4} & \multicolumn{1}{c|}{2} & \multicolumn{1}{c|}{{\color[HTML]{C0C0C0} x}} & \multicolumn{1}{c|}{{\color[HTML]{656565} 21}} \\ \hline
                \multicolumn{1}{c|}{{\color[HTML]{9B9B9B} \textit{colsums}}} & \multicolumn{1}{c|}{{\color[HTML]{656565} 420}} & \multicolumn{1}{c|}{{\color[HTML]{656565} 183}} & \multicolumn{1}{c|}{{\color[HTML]{656565} 153}} & \multicolumn{1}{c|}{{\color[HTML]{656565} 75}} & \multicolumn{1}{c|}{{\color[HTML]{656565} 26}} & \multicolumn{1}{c|}{\textit{857}} \\ \cline{2-7} 
                \multicolumn{1}{l}{} & \multicolumn{1}{l}{} & \multicolumn{1}{l}{} & \multicolumn{1}{l}{} & \multicolumn{1}{l}{} & \multicolumn{1}{l}{} & \multicolumn{1}{l}{} \\
                \multicolumn{1}{c|}{} & \multicolumn{1}{c|}{\textbf{Luminal A}} & \multicolumn{1}{c|}{\textbf{Luminal B}} & \multicolumn{1}{c|}{\textbf{Basal-like}} & \multicolumn{1}{c|}{\textbf{HER2-enriched}} & \multicolumn{1}{c|}{\textbf{Normal-like}} & {\color[HTML]{9B9B9B} \textit{rowsums}} \\ \hline
                \multicolumn{1}{c|}{\textbf{Lobular}} & \multicolumn{1}{c|}{114} & \multicolumn{1}{c|}{10} & \multicolumn{1}{c|}{3} & \multicolumn{1}{c|}{5} & \multicolumn{1}{c|}{11} & \multicolumn{1}{c|}{{\color[HTML]{656565} 143}} \\ \hline
                \multicolumn{1}{c|}{\textbf{Ductal}} & \multicolumn{1}{c|}{270} & \multicolumn{1}{c|}{158} & \multicolumn{1}{c|}{135} & \multicolumn{1}{c|}{68} & \multicolumn{1}{c|}{13} & \multicolumn{1}{c|}{{\color[HTML]{656565} 644}} \\ \hline
                \multicolumn{1}{c|}{{\color[HTML]{000000} \textbf{LobDuctal}}} & \multicolumn{1}{c|}{17} & \multicolumn{1}{c|}{6} & \multicolumn{1}{c|}{1} & \multicolumn{1}{c|}{{\color[HTML]{C0C0C0} x}} & \multicolumn{1}{c|}{{\color[HTML]{C0C0C0} x}} & \multicolumn{1}{c|}{{\color[HTML]{656565} 24}} \\ \hline
                \multicolumn{1}{c|}{\textbf{Metaplastic}} & \multicolumn{1}{c|}{{\color[HTML]{C0C0C0} x}} & \multicolumn{1}{c|}{{\color[HTML]{C0C0C0} x}} & \multicolumn{1}{c|}{7} & \multicolumn{1}{c|}{{\color[HTML]{C0C0C0} x}} & \multicolumn{1}{c|}{{\color[HTML]{C0C0C0} x}} & \multicolumn{1}{c|}{{\color[HTML]{656565} 7}} \\ \hline
                \multicolumn{1}{c|}{\textbf{Mucinous}} & \multicolumn{1}{c|}{8} & \multicolumn{1}{c|}{4} & \multicolumn{1}{c|}{{\color[HTML]{C0C0C0} x}} & \multicolumn{1}{c|}{{\color[HTML]{C0C0C0} x}} & \multicolumn{1}{c|}{{\color[HTML]{C0C0C0} x}} & \multicolumn{1}{c|}{{\color[HTML]{656565} 12}} \\ \hline
                \multicolumn{1}{c|}{\textbf{Other}} & \multicolumn{1}{c|}{11} & \multicolumn{1}{c|}{5} & \multicolumn{1}{c|}{7} & \multicolumn{1}{c|}{2} & \multicolumn{1}{c|}{2} & \multicolumn{1}{c|}{{\color[HTML]{656565} 27}} \\ \hline
                \multicolumn{1}{c|}{{\color[HTML]{9B9B9B} \textit{colsums}}} & \multicolumn{1}{c|}{{\color[HTML]{656565} 420}} & \multicolumn{1}{c|}{{\color[HTML]{656565} 183}} & \multicolumn{1}{c|}{{\color[HTML]{656565} 153}} & \multicolumn{1}{c|}{{\color[HTML]{656565} 75}} & \multicolumn{1}{c|}{{\color[HTML]{656565} 26}} & \multicolumn{1}{c|}{\textit{857}} \\ \cline{2-7} 
                \multicolumn{1}{l}{} & \multicolumn{1}{l}{} & \multicolumn{1}{l}{} & \multicolumn{1}{l}{} & \multicolumn{1}{l}{} & \multicolumn{1}{l}{} & \multicolumn{1}{l}{} \\
                \multicolumn{1}{c|}{} & \multicolumn{1}{c|}{\textbf{stage 1}} & \multicolumn{1}{c|}{\textbf{stage 2}} & \multicolumn{1}{c|}{\textbf{stage 3}} & \multicolumn{1}{c|}{\textbf{stage 4}} & \multicolumn{1}{c|}{\textbf{unknown}} & {\color[HTML]{9B9B9B} \textit{rowsums}} \\ \hline
                \multicolumn{1}{c|}{\textbf{Lobular}} & \multicolumn{1}{c|}{14} & \multicolumn{1}{c|}{78} & \multicolumn{1}{c|}{49} & \multicolumn{1}{c|}{{\color[HTML]{C0C0C0} x}} & \multicolumn{1}{c|}{2} & \multicolumn{1}{c|}{143} \\ \hline
                \multicolumn{1}{c|}{\textbf{Ductal}} & \multicolumn{1}{c|}{118} & \multicolumn{1}{c|}{368} & \multicolumn{1}{c|}{129} & \multicolumn{1}{c|}{11} & \multicolumn{1}{c|}{18} & \multicolumn{1}{c|}{644} \\ \hline
                \multicolumn{1}{c|}{\textbf{LobDuctal}} & \multicolumn{1}{c|}{5} & \multicolumn{1}{c|}{11} & \multicolumn{1}{c|}{7} & \multicolumn{1}{c|}{{\color[HTML]{C0C0C0} x}} & \multicolumn{1}{c|}{1} & \multicolumn{1}{c|}{24} \\ \hline
                \multicolumn{1}{c|}{\textbf{Metaplastic}} & \multicolumn{1}{c|}{2} & \multicolumn{1}{c|}{4} & \multicolumn{1}{c|}{1} & \multicolumn{1}{c|}{{\color[HTML]{C0C0C0} x}} & \multicolumn{1}{c|}{{\color[HTML]{C0C0C0} x}} & \multicolumn{1}{c|}{7} \\ \hline
                \multicolumn{1}{c|}{\textbf{Mucinous}} & \multicolumn{1}{c|}{3} & \multicolumn{1}{c|}{5} & \multicolumn{1}{c|}{4} & \multicolumn{1}{c|}{{\color[HTML]{C0C0C0} x}} & \multicolumn{1}{c|}{{\color[HTML]{C0C0C0} x}} & \multicolumn{1}{c|}{12} \\ \hline
                \multicolumn{1}{c|}{\textbf{Other}} & \multicolumn{1}{c|}{6} & \multicolumn{1}{c|}{15} & \multicolumn{1}{c|}{5} & \multicolumn{1}{c|}{1} & \multicolumn{1}{c|}{{\color[HTML]{C0C0C0} x}} & \multicolumn{1}{c|}{27} \\ \hline
                \multicolumn{1}{c|}{{\color[HTML]{9B9B9B} \textit{colsums}}} & \multicolumn{1}{c|}{148} & \multicolumn{1}{c|}{481} & \multicolumn{1}{c|}{195} & \multicolumn{1}{c|}{12} & \multicolumn{1}{c|}{21} & \multicolumn{1}{c|}{\textit{857}} \\ \cline{2-7} 
                \end{tabular}%
                }
                \end{table}

        
        
   
    
    Another way of exploring the structure within a dataset and the relevance of different classification conventions is to perform sample clustering and visualise it as a heatmap. Figure \ref{fig:heatmap1k} shows the results of performing clustering on top 1000 highest variance genes. Variance was computed for each row (gene) in cancer samples only, and the top 1000 were selected to be used for clustering, as the genes with the most differences are of interest. The three main effect groups are shown above the heatmap as colour bars, which enables side-by-side comparisons.     
    
    % heatmap 1000
            \begin{figure}[!h]
            \centering
            \includegraphics[scale=0.8]{heatmap_top1k.png}
            \caption{Heatmap with clustering on top 1000 highest variance genes (calculated on cancer samples only). The data is clustered by columns (samples) and rows (genes), with dendograms showing how clusters are formed. The colour bars above the heatmap (PAM50, stages, morphology) show the subgroups to which each sample belongs, colour-coded according to the legend on the right. The colours of heatmap represent gene expression intensity according to the scale (high expression - dark blue, low expression - light yellow). The data was clustered with Euclidean distance and average linkage. }
            \label{fig:heatmap1k}
            \end{figure}
    
    First and foremost, the unsupervised clustering was able to form two main clusters: Luminal and non-Luminal samples. Cluster on the right contains the majority of Luminal A and Luminal B samples according to the top colour bar (PAM50). The left hand side cluster contains Basal-like, HER2-enriched, and normal samples. Normal-like subtype is scattered across the entire dendogram, with more notable appearance inbetween Luminal A and normal samples, as expected. The Basal-like cluster is well-defined, which highlights this subtype difference from the rest. The Luminal types are mixed as anticipated. Overall, clustering is in agreement with the PCA results for PAM50 subtypes. \\    
    With regards to stages and morphology colour bars, it can be observed that clustering, much like PCA, was not able to find any distinct major patterns among their subgroups. Again, the sizes of the subgroups may partially be responsible for that. It is a challenge to spot the minority-sized subgroups on the colour bar, and very clear to see how Ductal carcinoma dominates the dataset (light blue, bottom bar). The colour bar of stages also shows no distinct patterns. 
    
    \newpage
    However, another idea worth investigation is to test how data will cluster if sub-subgroups averages are taken. As stages of cancer progression are of high interest in this project, the averages of each PAM50 subtype at each stage were taken and clustered. Figure \ref{fig:dendogram} shows the resulting dendogram. 
    
        % dendogram
            \begin{figure}[!h]
            \centering
            \includegraphics[scale=0.45]{dendogram2.png}
            \caption{Dendogram showing clustering of PAM50 subtypes within stages (averages per stage per subtypes)}
            \label{fig:dendogram}
            \end{figure}
    
    Remarkably, all subtypes form distinct branch clusters (expect HER2-enriched). Normal and Normal-like samples are the outgroup, understandably, and then Basal-like subtype is also an outgroup to the rest of samples. Luminal types and HER2-enriched are separated to well-defined branches. It is important to note that each subtype cluster has stage 4 branch as an outgroup, with the exception of HER2-enriched stage4, which is only made up of 2 samples perhaps leading to an unstable average. Seeing stage 4 as an outgroup is an important observation, as it is the most different and severe (due to metastasis) stage of cancer progression. \\
    
    Lastly, seeing the samples cluster primarily by PAM50 and only then by stage in Figure \ref{fig:dendogram}, is yet another evidence to what was observed with PCA and heatmap clustering. PAM50 is the main classification effect of breast cancer samples. When using other classification methods, PAM50 subtypes have to be taken into account. For example, comparing two morphologies might not produce truthful results if their PAM50 subtype sample composition is the main driving effect of differences. Therefore, the heterogeneity of groups has to be taken into account, and the downstream analysis has to be done on sub-subgroups.  
    
     

    % HCL implementing the Euclidean distance metric and the average linkage
    % distance method to analyse the same data used to generate the PCA in figure determines that the data clusters into two predominant groups — one comprising
    % all cancer samples and the other consisting of all control (healthy) samples. This
    % indicates that the cancer transcriptomic profiles are more similar to each other
    % than any of the control samples. These findings are reflected by those obtained
    % from the PCA.



    
 
    
    \newpage
    \subsection{Cofactors and batch effects}
    
    Morphology, cancer stages, and PAM50 molecular subtype are the expected main sources of variation in the dataset, namely biological variation. However, in a large and heterogeneous dataset such as TCGA-BRCA there are potentially several other sources of variation that may be contributing to the differences in gene expression between samples. Having sufficient clinical and meta annotation for the present samples allows us to explore potential sources of technical variation.
    %AGE is a standard covariate
    Among the available annotations, the most interesting ones are the patient age group, year sample was taken, and tissue source site. These were explored with PCA to visualise any outlying subgroups that may be creating an unwanted batch effect.
    Exploring the samples by year may identify significant differences between years, as the lab protocols for sample processing would have changes in the given years (1988-2013). Such differences could also be observed between the source sites, i.e. clinical institutions and research centers where the original samples were handled. 

    One-dimensional PCA plots were used to visualise variation over the years and across the sample sites. As there are many (>20) subgroups in both year ans site data, PCA scatter plots are not helpful for spotting outlying subgroups. 
    
    Figure \ref{fig:1dpcayear} shows the first 6 PCs of PCA of the year data. Each boxplot represents samples taken in a single year, with individual years shown as gitter-dots to visualise each year group approximate size and sample variation within it (according to y-axis). The boxplots are shown for years in sequential order (1988-2013). 
        
            % 1D PCA YEAR
            \begin{figure}[!h]
            \centering
            \includegraphics[scale=0.45]{1dpcayear.png}
            \caption{One-dimensional PCA plots showing variance along PC1-PC6 for all samples (cancer and normal) taken in years 1988-2013. The points over each box plot represent samples and their variation on the y-axis scale, to give a better indication of year group size and sample variation within it}
            \label{fig:1dpcayear}
            \end{figure}
    
    
    Overall, it is evident that there is a lot of variation between the years. This is especially the case for the earlier years, however, this is partially due to low number of samples taken in those years.  The variation stabilises after year 2005, which can be explained by the fact that this is when the pilot of TCGA project started and the universal protocols were introduced. Also it can be seen that the majority of samples were collected after that time point. Another aspect worth noticing is that samples in years 1997-2000 appear to be quite different to previous and following years (PC1). This could have had a very serious batch effect on the downstream analyses, and had to be explored further.\\
    
    As established in the previous section, we expect PAM50 subtypes to be the main point of interest for further investigation, therefore, it was crucial to check the sample composition of every year group in terms of PAM50 subtypes to make sure that, for example, a particular subtype was *not*  collected in only one year, which would be confounding the dataset exploration in that context. Figure \ref{fig:baryear} shows the stacked barplot of all samples organised by year.  Each bar shows the total count of samples in a year, and the proportion of samples of each PAM50 subtype are coloured according to the legend. 
    
    
             % bar per year
            \begin{figure}[!h]
            \centering
            \includegraphics[scale=0.7]{bar_year.png}
            \caption{Stacked barplots of sample counts per year (1988-2013). The coloured proportions represent PAM50 subtypes samples within each year group. }
            \label{fig:baryear}
            \end{figure}
    
    As already noted from 1D PCA plot fro year data, the total number of samples in the earlier years is marginally smaller than in the last decade of given years. Overall, the distribution of PAM50 subtypes in each year is well-balanced,  although some years inevitably have no normal samples and/or samples of a specific subtype. On the whole, the dataset is not confounded by a particular year-subtype combination.  Moreover, visualising data in this way has shed light on the abnormality in years 1997-2000 seen in 1D PCA plot. It can be seen in Figure \ref{fig:baryear} that these years have about 50\% composition of normal samples in them, which is not the case for all other years. Having such high proportion of normal samples in them, makes their mean seem different. Appendix \ref{} shows a 1D PCA plot for cancer samples only, confirming this possible explanation, i.e. years 1997-2000 do not stand out from the rest if only cancer samples are included in the analysis. \\
    
        %sites
    \newpage
    As with year data, the sample source data also had to be explored and variation among sources assessed. Figure \ref{fig:1dpcatss} shows a 1D PCA plot analogous to the one showing variation for samples grouped by year. This plot shows variation for all samples grouped by samples source site (24 sites in total).\\  
            
            % 1D PCA YEAR
            \begin{figure}[!h]
            \centering
            \includegraphics[scale=0.45]{1dpcatss.png}
            \caption{One-dimensional PCA plots showing variance along PC1-PC6 for all samples (cancer and normal) from different source sites (two letter abbreviations; NA - site unknown). The points over each box plot represent samples and their variation on the y-axis scale, to give a better indication of year group size and sample variation within it}
            \label{fig:1dpcatss}
            \end{figure}

    It is clear that the variation between source sites exists, as anticipated. As with year data, some source groups have very few samples, which makes the variation between them and large groups more prominent. One source site that really stands out (PC1) is BH (red). A possible explanation to this could have been that the sample handling protocol at this site is very different  from the rest, or, as it turned out, this differences has the same cause as the difference seen in year data.\\
    
              % bar per tss
            \begin{figure}[!h]
            \centering
            \includegraphics[scale=0.5]{bar_tss.png}
            \caption{Stacked barplots of sample counts per source site (alphabetic order). The coloured proportions represent PAM50 subtypes samples within each year group. }
            \label{fig:bartss}
            \end{figure}
    
    The source site data was visualised as stacked bar plots to inspect sample subtype variety coming from each site, Figure \ref{fig:bartss}. As observed on the 1D PCA plot, some sites have very few samples coming from them. The larger groups are relatively well-balanced in terms of their PAM50 subtype composition, with the exception of Normal-like, which comes only from 7 sites. This is, however, expected, as there are only 26 Normal-like samples in total. Another observation, which may be of bigger concern, is that the majority of normal samples come from one site, BH. This should not affect downstream analyses, as we are interested in differences between cancer subgroups. But seeing this explains the abnormal variation seen in 1D PCA plot for this site. As with year data, this one site is ~50\% composed of normal samples, making it appear very different from the rest as a group. If analysis is done only on cancer samples, this is no longer the case (Appendix \ref{}). 
    
  
    
    

    
    %age groups
    
    \subsection{Summary}
\newpage
\section{Looking for Autophagy signatures}

    \subsection{Differential Expression Analysis}
    
    Differential expression (DE) analysis was performed on the final dataset which included 857 tumour and 112 normal samples. After all filtering steps the dataset included 15784 genes, 1090 of which were autophagy-related genes. 

    Limma-voom method was used to quantify differential expression between the samples representing subgroups of the main breast cancer classification methods (PAM50 molecular subtypes, cancer stages, tumour morphology). The subgroups are described in Section XX and the number of samples per subgroup in the final dataset are presented in Tables \ref{table:morphstage} and \ref{table:pam50counts}. Figure \ref{fig:deaworkflow} shows the outline of DE analysis workflow. 

       
    % DEA workflow
            \begin{figure}[!h]
            \centering
            \includegraphics[scale=0.19]{dea_workflow.png} 
            \caption{Differential Expression analysis workflow for the TCGA-BRCA dataset. The main classification groups/effects (PAM50, stages, morphology) as well as additional cofactors (age group, sample source site and year) were included in the models for differential expression testing. The Venn diagram shows the six tested models; the outer sections show the three main effect models, and the overlapping 'Groups' show the three models of combined main effects. In total, $\approx 400$ contrasts across the six models were made to test for DE expression. For enrichment analysis, three gene sets were used (all autophagy genes (1090), core only (155), and TFs only (97)). Enrichment was tested in all DE genes, and also in the groups of up- or downregulated genes only. In total, 400x3x3 tests were made prior to downstream analysis.  }
            \label{fig:deaworkflow}
            \end{figure}

        \subsubsection{Main Effect Models}
        
        Each of the main sample classifications groups (PAM50, stages, morphology) were used as the main effect in individual models for DE testing. For each main effect, all combinations of subgroups and normal samples were included in the contrast matrix. In this way, for each PAM50 subtype (or each stage and morphology) differentially expressed genes were identified compared to the rest of subtypes and the normal samples. All models included the age group covariate as well as the two sources of batch effect: year and site. The raw numbers of differentially expressed genes in each contrast are available from Appendix X.

        Differential expression testing based on PAM50 subgroups has detected a fairly large number of differentially expressed genes (DEGs) in each comparison (min 5\% - Luminal A vs Luminal B, max 31\% - Basal-like vs Normal). Understandably, more of DEGs were found in contrasts with normal samples, with a noticeably larger proportion being downregulated in cancer samples. The proportions of DE-regulated genes between PAM50 subtypes ranged between 5\% and 22\% (Luminal A vs Luminal B and Luminal A vs Basal-like), and a subtype vs normal 12\% (Normal-like) to 31\% (Basal-like). 
        
        Differential expression testing based on cancer stages showed a considerable number of differentially expressed genes between stages and normal samples, $\approx 23\%$ at each stage. Interestingly, the analysis has found only very few genes to be DE between individual stages.
        
        Differential expression testing based on tumour morphology showed that the most DEGs-rich contrasts are again the ones with the normal samples. For morphology types the proportion of DEGs compared to normal varied between 21\% and 28\%. The two main morphologies, Lobular and Ductal carcinomas,  had only 6\% (924) DEGs between them. When both were tested with the mixed Lobular+Ductal morphology, both had less than 20 DEGs, meaning that this morphology is too much of a blend between the two separate ones to detect differential expression. Metaplastic and Mucinous morphologies were found to have the most DEGs with Lobular morphology (15\% and 13\%), and less with Ductal (6\% and 9\%) and Lobular+Ductal (10\% and 6\%) morphologies. The largest number of DEGs between two morphologies was between Metaplastic and Mucinous - 19\%. 
        
        Overall, large numbers of differentially expressed genes between PAM50 subtypes is an expected and reassuring result, as it is known from exploratory analysis that the differences between them is the strongest source of variation the breast cancer. Not seeing many differentially expressed between stages can be due to heterogeneity of the dataset, which is also probably contributing to differential expression in morphology. As noted previously with the PCA of morphology, the two types that show a clear separation (Metaplastic and Mucinous) are likely to be driven so by their exclusivity of PAM50 sample composition. Therefore, a large number of differentially expressed genes between them is also a likely consequence of that. 

        \subsubsection{Combined Effect Models}
        
        The reduce the effect of data heterogeneity in the differential expression analysis, additional models with combined groups effect were tested. Three models were created: PAM50 + stage (Group1), morphology + stage (Group2), PAM50 + morphology (Group3), as shown in Venn diagram in Figure \ref{fig:deaworkflow}. This analysis setup allowed testing for differential expression in more precisely defined groups,  for example, Group1 model is able to quantify expression between stages of a specific subtype, i.e. between stages 1 and 2 of Luminal A. Again, all models included the age group, year, and source site as cofactors.

        In general, a large number of separate PAM50 subtype-specific differentially expressed genes were found at individual stages and in specific morphologies. As the number of contrasts for each Group is around a hundred, the details will be omitted. 
        
        Differential expression analysis with Group1 model (PAM50 + stage) has shown that all subtypes at all stages have a large number of DEGs compared to normal samples. Between different stages of a single subtype, only Basal-like and Luminal B have a reasonable number of DE genes. Contrasts between the same stages of different PAM50 subtypes have shown that Luminal A and Basal-like have the largest number of differentially expressed genes consistently at all stages. Interestingly, at stage 4 there is the smallest numbers of differentially expressed genes between subtypes. 
        
        Differential expression analysis with Group2 model (stage + morphology) has shown that Ductal morphology has the largest number of DEGs at all stages.  Also, Ductal is the only morphology that contains stage 4 samples in this dataset, so the comparisons for the rest of morphologies were made based only on first three stages. The stage-wise comparison of Lobular and Ductal carcinomas has shown that the number of DEGs between them increases with stage. This seems reasonable, as both of them have a decent number of samples per stage, and also both are represented by a good mix of PAM50 subtypes (Table \ref{table:counts}). This is, however, not the case for contrasts involving Mucinous and Metaplastic types. As mentioned before, their PAM50 composition is restricted to Luminal and Basal subtypes, respectively, which makes all comparisons made with them dubious. The DE results between the two have shown a very large number of differentially expressed genes, particularly at stage 2. There, the comparison being made is between 4 and 5 samples, which are exclusively Basal-like and Luminal A. As noted earlier, the two subtypes show the highest number of DEGs on the all-samples level, and this compelling difference is maintained at the few-samples level too. 
        
        Differential expression analysis with Group3 model (PAM50 + morphology) is of interest mostly only for Lobular and Ductal morphologies, as they contain enough samples to represent each PAM50 subtype (Table \ref{table:counts}). Both have a large number of differentially expressed genes between all subtypes, with Ductal carcinoma having a distinctly large number for the contrast between Luminal A and Basal-like.  

        \subsubsection{Enrichment Analysis}
        
        Ultimately, we are interested not in the numbers of differentially expressed genes in each contrast per se, but in whether these subsets of DEGs are enriched for autophagy-related genes. As shown in the diagram in Figure \ref{fig:deaworkflow}, groups of up-, down-, and both direction regulated genes were tested for enrichment. Three separate gene sets were used in the Fisher’s test - all autophagy genes, only autophagy core genes, and only autophagy-related transcription factors (TFs). The significance of autophagy enrichment was evaluated based on the cut-off for adjusted p-value ($<0.05$) and the odds-ratios score ($>1$). The results significant as non-adjusted p-value were also considered.

        The enrichment results have, however, largely been unfruitful, particularly on the differential expression results from the individual main effect models. The raw data is shown in Appendix X2. 
        
        For the individual main effect models (only PAM50, stage, or morphology) none of the subsets of differentially expressed genes detected by the contrast tests have been assigned an odds-ratio $>1$ and simultaneously been significant at adjusted p-value level. There were several contrasts enriched for TFs but only at non-adjusted p-value - downregulation compared to normal:
        
        \begin{itemize}
            \item Luminal A, Luminal B, and Basal-like  (both directions in Luminal B)
            \item Stage 2, stage 3, and stage 4 
            \item Metaplastic, Mucinous, and Lobular+Ductal mixed \\
        \end{itemize}
        
    
        When using ‘Group’ models, i.e. combining two effects, some contrasts were found enriched at acceptable level. In Group1, ‘all autophagy’ genes were enriched in the downregulated genes at stage 4 of Basal-like compared to stage 4 of Luminal A and Luminal B, and also to normal samples. At non-adjusted p-value level, there were 13 contrasts enriched for TFs downregulated in a cancer type vs normal. Among these, there was stage 1 of every PAM50 subtype, all stages of Basal-like subtype (Appendix X2). 
         
        In Group2, autophagy TFs were enriched in the downregulated genes in every morphology compared to normal, at varying stages. There was also a strong enrichment in TFs that are differentially expressed between Lobular and Ductal carcinomas at stage 1. At non-adjusted p-value level, there was a large number of contrasts enriched for autophagy TFs, mainly in the downregulated in cancer vs normal fashion. Among these are all stages of Ductal and Mucinous types, as well as stages 1 and 3 of Metaplastic. Interestingly, Lobular carcinoma did not appear in this context at all. Nonetheless, groups of upregulated genes in Lobular compared to Metaplastic (stage 3) and Mucinous (stage 2) were enriched for autophagy (Appendix X3). 
        
        In Group3, ‘all autophagy’ genes were enriched in the downregulated genes in Basal-like samples compared to Luminal A and Normal-like samples (all of Lobular morphology) and normal samples. When looking at non-adjusted p-value, Luminal B can also be added to this pattern. For non-adjusted p-value enrichment for TFs, there were multiple enriched contrasts, again including several instances of downregulation in cancer vs normal (particularly Basal-like and Luminal B), DE between morphologies within Basal-like subtype, and again similar for Lobular carcinoma as for ‘all autophagy’.
        
        Overall, it seems that using the combined main effects models allows to detect more autophagy enrichment in the differential expression results. As only specific combinations of samples are considered, the heterogeneity is decreased, allowing for a stronger differential expression signal, i.e. more genes are reported. However, it still cannot be claimed that enrichment results attained from the combined models provide strong evidence of a particular group of samples to be enriched for autophagy. The main observations are:
        
        \begin{itemize}
            
        \item The main recurrent trend is enrichment for autophagy in genes downregulated in cancer (different subgroups and their combination)
        \item There is a consistent enrichment for autophagy TFs in genes downregulated in separate stages/morphologies (at adjusted and non-adjusted p-value level)
       \item No enrichment is found between stages. Neither when stages is the main effect, and also not when looking at stages in separate PAM50 subtypes or morphologies
       \item Samples that belong to subgroups Basal-like subtype, Lobular morphology, stage 4, seem to often appear among the enriched groups. This could have been an interesting target, but this data set does not contain enough samples in these subgroups to make the speculation sound. Stage 4 samples in the dataset are only of Ductal morphology. Samples that are Basal-like and are of Lobular morphology, there are only 3 ( Table \ref{table:counts}).  

        
        \end{itemize}
        \subsubsection{Summary}
        

\newpage
    \subsection{Soft clustering}
    
    
        This section will talk about the clustering setup: how many clusters, what input gene list and why (DE genes to account for BE), membership score etc.
        
        Then explain that we are interested in clustering by stage - to see expression patterns, and then enrich clusters for autophagy.
        
        \subsubsection{All samples}
        
        Results from clustering on all subtypes: autophagy enriched in downregulated cluster. 
        
        
        \subsubsection{Subtype specific}
        
        Results from clustering on each PAM50 subtype separately, enrichment.

        
        \subsubsection{Summary}

        
    
    
    
    
    
    %identify distinct transcriptome clusters
    %classify genes into groups
    
    %     The expression profiles of these 348 genes were sorted into 5 clusters by using the fuzzy c-means algorithm implemented in the R package Mfuzz (20). The Mfuzz package uses a “fuzzy” clustering approach to compute the strength of the association of each gene with the central behavior of the cluster, referred to as a “fuzzy score.” The fuzzy scores for each gene sum to 1. This allows one to define a set of genes that associate with each cluster with a high probability, while other genes may not have a high score for association with a single cluster but may instead have scores that allow partial placement in more than one cluster. This approach is more flexible and less sensitive to experimental noise than traditional “hard” clustering, where each gene must belong to one and only one cluster.
 
% Table S1 in the supplemental material shows each of the 348 genes assigned to the 5 clusters and their individual fuzzy scores. The median fuzzy score for each cluster was >0.98, and the average fuzzy score was >0.9, indicating strongly correlated behaviors of the genes within each cluster with respect to changes in growth rate and substrate. Thus, each cluster has a core set of genes with very high fuzzy scores (i.e., close to 1) which indicate that their collective expression behavior is strongly correlated with that of other members of the cluster with respect to their changes in growth rate and substrate. Moreover, the lack of significant overlap of genes between the clusters also supports the unique characteristics of the clusters and their embedded genes.



 
% To overcome the limitations of hard clustering, soft clustering can be applied o
% ering several advantages to researchers [1, 2]. First, it generates accessible internal cluster structures, i.e. it indicates how well corresponding clusters represent genes/proteins. This additional information can be used for a rened search for regulatory elements. Second, the overall relation between clusters, and thus a global clustering structure, can be dened. Additionally, soft clustering is more noise robust and a priori pre-ltering of genes can be avoided. This prevents the exclusion of biologically relevant genes/proteins from the data analysis.



